\documentclass[10pt]{beamer}
\usepackage[utf8]{inputenc}

%\usepackage{multirow,rotating}
\usepackage{color}
%\usepackage{hyperref}
%\usepackage{tikz-cd}
%\usepackage{array}
%\usepackage{siunitx}
%\usepackage{mathtools,nccmath}%
%\usepackage{etoolbox, xparse} 

%\usetheme{CambridgeUS}
%\usecolortheme{dolphin}
\usetheme{Warsaw}
\usecolortheme{whale}

% set colors
\definecolor{myNewColorA}{RGB}{26, 47,150}
\definecolor{myNewColorB}{RGB}{26, 47, 233}
\definecolor{myNewColorC}{RGB}{26, 47, 150} % {130,138,143}
\definecolor{myGrey}{RGB}{220,220,240}
\definecolor{dkgreen}{RGB}{100,150,100}

%\setbeamercolor*{palette primary}{bg=myNewColorC, fg = white}
%\setbeamercolor*{palette secondary}{bg=myNewColorB, fg = white}
%\setbeamercolor*{palette tertiary}{bg=myNewColorA, fg = white}
%\setbeamercolor*{titlelike}{fg=myNewColorA,bg=myGrey}
%\setbeamercolor*{title}{bg=myNewColorA, fg = white}
%\setbeamercolor*{item}{fg=myNewColorA}
%\setbeamercolor*{caption name}{fg=myNewColorA}

%\setbeamercolor{block title}{fg=white,bg=myNewColorA}
%\setbeamercolor{block body}{use=titlelike,bg=titlelike.bg,fg=black}
%\setbeamercolor{frametitle}{bg=myGrey}

\usefonttheme{professionalfonts}
\usepackage{natbib}
\usepackage{hyperref}
%------------------------------------------------------------
% \titlegraphic{\includegraphics[height=0.75cm]{ua_eng_logo.png}} 

% logo of my university


\titlegraphic{%
\includegraphics[width=3.5cm]{logo.jpeg}
}

\setbeamerfont{title}{size=\large}
\setbeamerfont{subtitle}{size=\small}
\setbeamerfont{author}{size=\small}
\setbeamerfont{date}{size=\footnotesize}
\setbeamerfont{institute}{size=\footnotesize}
\title[Informatique]{Cours d'Informatique: Introduction}%title
%\subtitle{ }%%subtitle
\author[Simon Halfon]{Simon Halfon; halfon@lsv.fr}%%authors

\institute{Lyc\'ee Saint-Louis}

\date{MP2I, 2022-2023}
%\date[\textcolor{white}{Conference Name, 2022}]
%{Conference full name\\
%Aug 29-30, 2022}

%% %------------------------------------------------------------
%% %This block of commands puts the table of contents at the 
%% %beginning of each section and highlights the current section:
%% %\AtBeginSection[]
%% %{
%% %  \begin{frame}
%% %    \frametitle{Contents}
%% %    \tableofcontents[currentsection]
%% %  \end{frame}
%% %}
%% \AtBeginSection[]{
%%   \begin{frame}
%%   \vfill
%%   \centering
%%   \begin{beamercolorbox}[sep=8pt,center,shadow=true,rounded=true]{title}
%%     \usebeamerfont{title}\insertsectionhead\par%
%%   \end{beamercolorbox}
%%   \vfill
%%   \end{frame}
%% }
%% % ------Contents below------
%------------------------------------------------------------

\begin{document}

%The next statement creates the title page.
\frame{\titlepage}

\begin{frame}
  \frametitle{Qu'est-ce que l'Informatique ?}

  \begin{block}{Informatique $\neq$ Ordinateur !}
    ``Computer science is no more about computers than astronomy is about telescopes'' \\
    -- Hal Abelson.
    \end{block}
  \pause
  \vspace{0.5cm}

  \begin{itemize}
  \item N\'eologisme 1957 (Karl Steinbuch): Informatique = traitement automatique de l'information.
  \item Acad\'emie fran\c caise 1966: Informatique = science du traitement de l'information.
  \end{itemize}

  \pause
  \vspace{0.5cm}

  \begin{tabular}{ccccc}
    Traitement automatique & $\Rightarrow$ & Calcul & $\Rightarrow$ &
    \begin{tabular}{c} Algorithme \\ (Al-Khw\^arizm\^i) \end{tabular} \\
    Information & $\Rightarrow$ & Donn\'ees (Data) & $\Rightarrow$ &
    \begin{tabular}{c} encodage, repr\'esentation, \\ syntaxe \end{tabular}
  \end{tabular}

\end{frame}




% consider removing it if it's too redundant
%% \AtBeginSection[]
%% {
%%   \begin{frame}
%%     \frametitle{Table of Contents}
%%     \tableofcontents[currentsection]
%%   \end{frame}
%% }

%------------------------------------------------------------
% \section{Introduction}
\begin{frame}{Algorithme VS machine \`a calculs}


  \begin{itemize}
  \item La pascaline (1645): \\
    \begin{center}
      \includegraphics[width=5.0cm]{pascaline.jpeg}
      \end{center}

  \item Algorithme d'Euclide [Les Elements d'Euclide, $\sim$-300]: \\
    \begin{center}
      \includegraphics[width=5.0cm]{euclide.jpeg}
      \end{center}
    
    
  \end{itemize}

\end{frame}


\begin{frame}{Algorithmes pour humains}


  \begin{itemize}
  \item Al-Khw\^arizm\^i, ``L'Abrégé du calcul par la restauration et la comparaison'': m\'ethodes g\'en\'erales de r\'esolutions de probl\`emes courants.  \\
      \includegraphics[width=2.5cm]{ouvrage.jpeg}      

      \vspace{-1cm}
  \item Tables de logarithme: \includegraphics[width=6.0cm]{log.jpeg}

    
  \end{itemize}

\end{frame}




\begin{frame}{Encodage de l'information: les entiers naturels}


  \begin{itemize}
  \item Les v\'erit\'es math\'ematiques sont ind\'ependantes de la repr\'esentation choisie:
    \begin{itemize}
\item $5 = 1 + 1 + 1 + 1 + 1$
\item
      $(23,12)$ est l'unique solution du systeme
      $\left\{ \begin{array}{c} y = 2x - 34 \\ y = -3x + 81 \end{array}\right.$.
\end{itemize}

\pause
  \item Base (d\'ecimale, hexad\'ecimale, ...) = encodage compact des entiers.

      \begin{center}
      \includegraphics[width=3.0cm]{tablette.jpeg}
      \end{center}
\pause
    \item Compromis:
      \begin{itemize}
      \item Algorithmes de calculs plus compliqu\'es.
      \item Algorithmes de calculs plus rapides. 
      \end{itemize}
      
  \end{itemize}

\end{frame}



\begin{frame}{Conclusion}

  \begin{block}{Algorithmique (S2)}
    Probl\`eme: un \textcolor{purple}{traitement} d'une \textcolor{dkgreen}{information} \\
    Traduction: une \textcolor{purple}{op\'eration} sur un \textcolor{dkgreen}{type de donn\'ees} \\
    \vspace{0.5cm}
    Objectif: encoder l'\textcolor{dkgreen}{information} ET inventer un \textcolor{purple}{algorithme} \\
    qui calcule \textbf{correctement} cette op\'eration sur cet encodage et en optimisant la consommation des \textbf{ressources (temps, m\'emoire, ...)}.
  \end{block}

\vspace{0.5cm}
  \pause

  \begin{block}{Programmation = \'ecriture de programmes (S1)}
    Programme = impl\'ementation d'un algorithme pour \^etre ex\'ecut\'e par une machine (dans un langage bien fig\'e et propre \`a la machine).
    \end{block}

  
\end{frame}


\end{document}
