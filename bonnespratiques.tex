\documentclass{article}


\title{Bonnes Pratiques et Discipline de Programmation}
\author{Simon Halfon}


\begin{document}


\maketitle



\section{La forme}


\subsection{Indentation}

\subsection{Accolade}


\subsection{Noms des variables}


\section{Cahier des charges / Sp\'ecification}

\begin{itemize}
  \item Programmes principaux (qui r\'epondent \`a un \'enonc\'e):

Ces programmes doivent r\'epondre \`a une sp\'ecification pr\'ecise: format de l'entr\'ee, format et canal de sortie, propri\'e\t'e de la sortie (correction du programme).
La sp\'ecification minimale \'etant la signature (typage) de la fonction.

Il peut \^etre utile de v\'erifier que le format de l'entr\'ee est v\'erifi\'e, et lever une exception dans le cas contraire.

\item Programmes auxiliaires:
  Ils doivent \^etre le plus souvent accompagn\'es de commentaires indiquant les sp\'ecifications (signature, pr\'e-conditions, post-conditions).
\end{itemize}
  
Les parties cons\'equentes des programmes peuvent \^etre accompagn\'e de variant et d'invariants de boucles pour justifier de la terminaison et correction.




\section{Debugging}

\subsection{Messages d'erreurs courants}


\subsubsection{En C}

\subsubsection{En OCaml}

\subsection{Assert et Printf}

Les pr\'e-condition et invariants de boucles peuvent \^etre ins\'er\'es dans le code au moyen de la fonction assert. Cela permet de confirmer \`a l'ex\'ecution que les invariants th\'eoriques sont v\'erifi\'es. Une fois la phase de debugging, on peut d\'esactiver les assert afin de gagner en performance.

\subsection{GDB}




\section{Tester le code}


\section{Pour aller plus loin}


Plateforme d'auto-evaluation:
%% Caseine
%% Wims
%% France IOI
%% Fun MOOC:
%% https://www.fun-mooc.fr/fr/cours/python-3-des-fondamentaux-aux-concepts-avances-du-langage/
%% https://www.fun-mooc.fr/fr/cours/abc-du-langage-c/
%% OpenClassroom
%% Projet Euler
%% Code Wars et autre hackaton de Samy.

%% https://essok.learn-ocaml.org/


%% Plateforme apprentissage OCaml
%% https://ocaml.org/docs

